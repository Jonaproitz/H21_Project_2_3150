\documentclass[english,notitlepage]{revtex4-1}  % defines the basic parameters of the document
%For preview: skriv i terminal: latexmk -pdf -pvc filnavn



% if you want a single-column, remove reprint

% allows special characters (including æøå)
\usepackage[utf8]{inputenc}
%\usepackage[english]{babel}

%% note that you may need to download some of these packages manually, it depends on your setup.
%% I recommend downloading TeXMaker, because it includes a large library of the most common packages.

\usepackage{physics,amssymb}  % mathematical symbols (physics imports amsmath)
\include{amsmath}
\usepackage{graphicx}         % include graphics such as plots
\usepackage{xcolor}           % set colors
\usepackage{hyperref}         % automagic cross-referencing (this is GODLIKE)
\usepackage{listings}         % display code
\usepackage{subfigure}        % imports a lot of cool and useful figure commands
\usepackage{float}
%\usepackage[section]{placeins}
\usepackage{algorithm}
\usepackage[noend]{algpseudocode}
\usepackage{subfigure}
\usepackage{tikz}
\usetikzlibrary{quantikz}
% defines the color of hyperref objects
% Blending two colors:  blue!80!black  =  80% blue and 20% black
\hypersetup{ % this is just my personal choice, feel free to change things
    colorlinks,
    linkcolor={red!50!black},
    citecolor={blue!50!black},
    urlcolor={blue!80!black}}

%% Defines the style of the programming listing
%% This is actually my personal template, go ahead and change stuff if you want



%% USEFUL LINKS:
%%
%%   UiO LaTeX guides:        https://www.mn.uio.no/ifi/tjenester/it/hjelp/latex/
%%   mathematics:             https://en.wikibooks.org/wiki/LaTeX/Mathematics

%%   PHYSICS !                https://mirror.hmc.edu/ctan/macros/latex/contrib/physics/physics.pdf

%%   the basics of Tikz:       https://en.wikibooks.org/wiki/LaTeX/PGF/Tikz
%%   all the colors!:          https://en.wikibooks.org/wiki/LaTeX/Colors
%%   how to draw tables:       https://en.wikibooks.org/wiki/LaTeX/Tables
%%   code listing styles:      https://en.wikibooks.org/wiki/LaTeX/Source_Code_Listings
%%   \includegraphics          https://en.wikibooks.org/wiki/LaTeX/Importing_Graphics
%%   learn more about figures  https://en.wikibooks.org/wiki/LaTeX/Floats,_Figures_and_Captions
%%   automagic bibliography:   https://en.wikibooks.org/wiki/LaTeX/Bibliography_Management  (this one is kinda difficult the first time)
%%   REVTeX Guide:             http://www.physics.csbsju.edu/370/papers/Journal_Style_Manuals/auguide4-1.pdf
%%
%%   (this document is of class "revtex4-1", the REVTeX Guide explains how the class works)


%% CREATING THE .pdf FILE USING LINUX IN THE TERMINAL
%%
%% [terminal]$ pdflatex template.tex
%%
%% Run the command twice, always.
%% If you want to use \footnote, you need to run these commands (IN THIS SPECIFIC ORDER)
%%
%% [terminal]$ pdflatex template.tex
%% [terminal]$ bibtex template
%% [terminal]$ pdflatex template.tex
%% [terminal]$ pdflatex template.tex
%%
%% Don't ask me why, I don't know.

\begin{document}

\title{Title of the document}      % self-explanatory
\author{Your name(s) here}          % self-explanatory
\date{\today}                             % self-explanatory
\noaffiliation                            % ignore this, but keep it.


\maketitle 
    

\section*{Problem 1.}
    Given
    \begin{equation}
            \gamma \frac{d^2 u(x)}{d x^2}
        =   -Fu(x)
        \label{eq4}
    \end{equation}
    with the definition $\hat{x} = x/L$, such that
    \begin{equation*}
            \frac{d\hat{x}}{d x}
        =   \frac{1}{L}
        \implies 
            dx
        =   L d\hat{x}
    \end{equation*}
    Equation \ref{eq4} can be written
    \begin{equation*}
            \gamma \frac{d^2 u(\hat{x})}{L^2 dx^2}
        =   -Fu(\hat{x})
        \implies
            \frac{d^2 u(\hat{x})}{d \hat{x}^2}
        =   -\frac{FL^2}{\gamma}u(\hat{x})
        =   -\lambda u(\hat{x})
    \end{equation*}
    with $\lambda = FL^2/\gamma$.$\hfill\blacksquare$


\section*{Problem 2.}
    For an arbitrary composite matrix $A = BC$ the transpose of $A^T = (BC)^T = C^TB^T$.
    Hence for $\vec{w}_i = U\vec{v}_i$
    \begin{equation*}
            \vec{w}^T_i\vec{w}_j
        =   \vec{v}^T_iU^TU\vec{v}_j
        =   \vec{v}^T_i\vec{v}_i
        =   \delta_{i,j}
    \end{equation*}
    as $U^TU = U^{-1}U = I$

   
\end{document}
